\documentclass[a4paper, fleqn]{article}

\usepackage[utf8]{inputenc}
\usepackage[english]{babel}
\usepackage[T1]{fontenc}
\usepackage[bitstream-charter]{mathdesign}
\usepackage{booktabs}
\usepackage{amsmath}
\usepackage[a4paper, left=2.5cm, right=2.5cm, top=2cm, bottom=2.5cm]{geometry}

\title{Homework Exercise 2}
\author{COM (BA) Statistics - WS 2020}
\date{October 27, 2020}

\begin{document}
\maketitle
\thispagestyle{empty}

\noindent
\textbf{Due date:} November 5, 2020 \\
\textbf{Max. points:} 8 \\

\begin{enumerate}
  % data preparation
  \item Download the data file \textit{student\_questionnaire.csv} from Moodle. The data file contains all responses to the student questionnaire. 

  \vspace{1em}
  \item Import the data file in the statistcal software of your choice.\\
    \textbf{In SPSS:} Use the import functionality in \textit{File} $\rightarrow$ \textit{Open} $\rightarrow$ \textit{Data}.

  \vspace{1em}
  \item Describe all variables (columns) in the data.
  \begin{enumerate}
    \item Which possible values can the variables take?
    \item What is the scale level of the variables?
  \end{enumerate}
  \textbf{In SPSS:} Also set the correct variable ranges and scales in the variable view if necessary.  

  \vspace{1em}
  \item Calculate the handedness score. The handedness score is the measurement of a single latent variable which incorporates all measured behavioural indicators. A pure right-hander will have a handedness score of +1 and a pure left-hander will have a handedness score of -1.

  \begin{enumerate}
    \item Convert the variables \textit{handedness\_*} from unipolar to bipolar measurements by subtracting the neutral category. 
    \item Scale the variables \textit{handedness\_*} such that the most extreme categories are equal to -1 and +1.
    \item Calculate a new variable \textit{handedness\_score} as the arithmetic mean of all \textit{handedness} variables. 
  \end{enumerate}
  \textbf{In SPSS:} Transform existing variables or calculate new variables via  \textit{Transform} $\rightarrow$ \textit{Compute Variable} or \textit{Transform} $\rightarrow$ \textit{Recode into Same/Different Variables}. 

  \vspace{1em}
  % univariate descriptives
  \item Summarize the variables \textit{sex}, \textit{age}, \textit{height}, \textit{handspan} and \textit{handedness\_score} numerically and visually. 
  
  \vspace{0.5em}
  Are there any abnormalities in the data? 
  \begin{itemize}
    \item If yes, can you find an explanation for them?
    \item Are they fixable? If yes, fix the issues and document the changes. If no, delete the abnormal values and define them as missing data. 
  \end{itemize}
  What does the data tell you about the people in this course?  

\vspace{0.5em}
\textbf{In SPSS:} Missing values can be set in the variable view. 

  \vspace{1em}
\item How is \textit{height} associated with the variables \textit{handspan} and \textit{handedness\_score}? Calculate the appropriate descriptive statistics and visualize the associations in the data. Also provide a verbal interpretation of the results.

\vspace{0.5em}
\textbf{In SPSS:} To calculate correlations, use \textit{Analyze} $\rightarrow$ \textit{Correlate} $\rightarrow$ \textit{Bivariate}. For contingency tables use \textit{Analyze} $\rightarrow$ \textit{Descriptive Statistics} $\rightarrow$ \textit{Crosstabs}.

\end{enumerate}

\vspace{2em}
\textit{Don't forget to save the processed dataset!}

\end{document}

