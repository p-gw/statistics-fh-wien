\documentclass[a4paper, fleqn]{article}

\usepackage[utf8]{inputenc}
\usepackage[english]{babel}
\usepackage[T1]{fontenc}
\usepackage[bitstream-charter]{mathdesign}
\usepackage{booktabs}
\usepackage{amsmath}
\usepackage[a4paper, left=2.5cm, right=2.5cm, top=2cm, bottom=2.5cm]{geometry}

\title{Homework Exercise 1}
\author{COM (BA) Statistics - WS 2020}
\date{October 5, 2020}

\begin{document}
\maketitle
\thispagestyle{empty}

\noindent
\textbf{Due date:} October 12, 2020 \\
\textbf{Max. points:} 8 \\

\begin{enumerate}
  % first task
  \item Think about a research question you want to test empirically using a survey or questionnaire as your measurement instrument. With the information we covered in the lecture on research design, write a short report on how you would approach your research project. Make sure to cover the following questions: 
    \begin{itemize}
      \item How did you arrive at your hypothesis?
      \item How would you operationalize your variables of interest?
      \item What is your target population?
      \item How would you collect the data?
    \end{itemize}

  \vspace{1em}
  % second task
  \item Develop a short questionnaire (10-15 questions maximum) which might be used to answer your research question. Make use of the practical guidelines given in the lecture. When developing your questionnaire especially think about
    \begin{itemize}
      \item which response format(s) is/are appropriate for your application, and 
      \item which type(s) of questions your questionnaire should be based on. 
    \end{itemize}
\end{enumerate}

\vspace{2em}
\textit{Try to avoid the common pitfalls introduced in the lecture!}

\end{document}

