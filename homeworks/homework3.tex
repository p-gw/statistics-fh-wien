\documentclass[a4paper, fleqn]{article}

\usepackage[utf8]{inputenc}
\usepackage[english]{babel}
\usepackage[T1]{fontenc}
\usepackage[bitstream-charter]{mathdesign}
\usepackage{booktabs}
\usepackage{amsmath}
\usepackage[a4paper, left=2.5cm, right=2.5cm, top=2cm, bottom=2.5cm]{geometry}

\title{Homework Exercise 3}
\author{COM (BA) Statistics - WS 2020}
\date{November 9, 2020}

\begin{document}
\maketitle
\thispagestyle{empty}

\noindent
\textbf{Due date:} November 20, 2020 \\
\textbf{Max. points:} 8 \\

\begin{enumerate}
  % sets and probability
  \item In the 3-dimensional world we live in, there are a total of five shapes from which we can construct fair dice. 
    The \textit{octahedron}, for example, is a shape with a total of 8 identical faces. Assuming we number the faces of an octahedron from 1 - 8, and the dice is fair (rolling each face has the same probability),
  \begin{enumerate}
    \item Construct the set of all possible outcomes for a dice of this type.
    \item State the probability for each possible outcome (construct the probability mass function)
    \item Construct the sets for the following outcomes and calculate their probabilities.
      \begin{itemize}
        \item A roll showing a value less or equal to 3.
        \item A roll showing a value greater than 3.
        \item A roll showing a value between 4 and 7.
        \item A roll showing a value less than 4 or greater than 7.
      \end{itemize}
    \item Calculate the expected value and variance of this probability distribution. 
  \end{enumerate}

  % conditional probability
  \vspace{1em}
\item Assuming five percent in the population have high blood pressure. Out of the five percent with high blood pressure, 75\% consume alcohol regularly. 
  \begin{enumerate}
    \item Is high blood pressure independent from regular alcohol consumption?
    \item Calculate the probability that a person has high blood pressure and consumes alcohol regularly. 
  \end{enumerate}

  % Poisson distribution (theoretical)
  \vspace{1em}
  \item A company fabricates 500 units of a product per day. The probability that a fabricated product is defective is 0.015. What is the probability that on any given day, 
    \begin{enumerate}
      \item exactly 8 products are defective?
      \item more than 10 products are defective?
    \end{enumerate}

  % Normal distribution (theoretical)
  \vspace{1em}
  \item The intelligence quotient (IQ) in the population is assumed to follow a normal distribution with expected value $\mu = 100$ and standard deviation $\sigma = 15$. You complete an IQ test and receive a score of 114. 
    \begin{enumerate}
      \item What is the fraction of the population with an IQ score less than your result? 
      \item What is the fraction of the population with an IQ score between 86 and 114 (100 $\pm$ 14)?  
      \item What is the IQ score required to belong to the 5\% of most intelligent people in the population?
    \end{enumerate}

  % Binomial model
  \vspace{1em}
\item In the \textit{student questionnaire} we observed 26 students stating they are female out of a total number of 38 responses. Assuming this cohort is representative of all cohorts of the bachelor's program \textit{Corporate Communication}, estimate the probability that a student in this study program is female. Also, provide a 50\% confidence interval for the estimate. How is this confidence interval interpreted?

  % Normal model
  \vspace{1em}
  \item Assuming this cohort is a representative sample of students in this bachelor's program, estimate the average height in the student population and calculate a 87\% confidence interval for the estimate using the \textit{student questionnaire} data. 

    \vspace{0.5em}
    \textbf{In SPSS:} To calculate confidence intervals for the mean, use \textit{Analyze} $\rightarrow$ \textit{Compare Means} $\rightarrow$ \textit{One-Sample T Test} and select the appropriate coverage probability in \textit{Options}.
\end{enumerate}


\newpage
\noindent
\textbf{SPSS notes} 
\begin{itemize}
  \item \texttt{PDF.BINOM} provides the probability mass function for the binomial distribution
  \item \texttt{CDF.BINOM} provides the cumulative distribution function for the binomial distribution
  \item \texttt{PDF.POISSON} provides the probability mass function for the Poisson distribution
  \item \texttt{CDF.POISSON} provides the cumulative distribution function for the Poisson distribution
  \item \texttt{PDF.NORMAL} provides the probability mass function for the normal distribution
  \item \texttt{CDF.NORMAL} provides the cumulative distribution function for the normal distribution
  \item \texttt{IDF.NORMAL} provides the quantile function for the normal distribution
\end{itemize}

\noindent
\textbf{Excel notes} 
\begin{itemize}
  \item \texttt{BINOM.DIST} provides the probability mass function and cumulative distribution function for the binomial distribution
  \item \texttt{BINOM.INV} provides the quantile function for the binomial distribution
  \item \texttt{POISSON.DIST} provides the probability mass function and cumulative distribution function for the Poisson distribution
  \item \texttt{NORM.DIST} provides the probability density function and cumulative distribution function for the normal distribution
  \item \texttt{NORM.INV} provides the quantile function for the normal distribution
\end{itemize}

\noindent
\textbf{R notes} 
\begin{itemize}
  \item \texttt{dbinom} provides the probability mass function for the binomial distribution
  \item \texttt{pbinom} provides the cumulative distribution function for the binomial distribution
  \item \texttt{qbinom} provides the quantile function for the binomial distribution
  \item \texttt{dpois} provides the probability mass function for the Poisson distribution
  \item \texttt{ppois} provides the cumulative distribution function for the Poisson distribution
  \item \texttt{qpois} provides the quantile function for the Poisson distribution
  \item \texttt{dnorm} provides the probability density function for the normal distribution
  \item \texttt{pnorm} provides the cumulative distribution function for the normal distribution
  \item \texttt{qnorm} provides the quantile function for the normal distribution
\end{itemize}


\end{document}

