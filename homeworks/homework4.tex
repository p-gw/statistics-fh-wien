\documentclass[a4paper, fleqn]{article}

\usepackage[utf8]{inputenc}
\usepackage[english]{babel}
\usepackage[T1]{fontenc}
\usepackage[bitstream-charter]{mathdesign}
\usepackage{booktabs}
\usepackage{amsmath}
\usepackage[a4paper, left=2.5cm, right=2.5cm, top=2cm, bottom=2.5cm]{geometry}

\title{Homework Exercise 4}
\author{COM (BA) Statistics - WS 2020}
\date{November 23, 2020}

\begin{document}
\maketitle
\thispagestyle{empty}

\noindent
\textbf{Due date:} December 7, 2020 \\
\textbf{Max. points:} 8 \\

\begin{enumerate}
  % continuous predictor
  \item Using the \textit{student questionnaire data} fit a linear regression model to the data, predicting the \textit{height} from the \textit{handspan} of students.
  \begin{enumerate}
    \item Estimate the intercept and slope of the regression line and provide 95\% confidence intervals for the estimates.

    \vspace{0.5em}
    \textbf{In SPSS:} You can fit a linear regression via \texttt{Analyze} $\rightarrow$ \texttt{Regression} $\rightarrow$ \texttt{Linear}. 

    \item Check the assumptions of the regression model by creating a residual plot, displaying the fitted values against the residuals. 

    \vspace{0.5em}
    \textbf{In SPSS:} Save predicted values and residuals for plotting using the \texttt{Save} functionality in the linear regression module.

    \item Evaluate the goodness-of-fit of the regression model by calculating the coefficient of determination. Does the model do a good job of fitting the data?

    \item For a handspan of 20 centimeters, what is the 50\% prediction interval? How is this interval interpreted? 

    \vspace{0.5em}
    \textbf{In SPSS:} You can create predictions for new values by adding the value as new data (leave other variables missing) and save predicted values using the \texttt{Save} functionality in the linear regression module.
  \end{enumerate}

  % dichotomous predictor
  \vspace{1em}
  \item Using the \textit{student questionnaire data} fit a linear regression model to the data, estimating the sex difference in \textit{handspan}.
  \begin{enumerate}
    \item Create a new dummy coded variable which labels female students as 1 and 0 otherwise. 
    \item Estimate the group difference in handspan and provide 80\% confidence intervals for the estimates. What are the estimated means for female and non-female students? 
    \item Evaluate the goodness of fit for the model using the residual standard deviation. How is the residual standard deviation interpreted for this model? 

    \vspace{0.5em}
    \textbf{In SPSS:} The residual standard deviation is provided in the output table \texttt{Model Summary} as \texttt{Std. Error of the Estimate}.  

    \item If we were to predict values for the next student cohort, what are the intervals of handspans that contains 95\% of new observations of female and non-female students? 
  \end{enumerate}

\end{enumerate}

\end{document}
